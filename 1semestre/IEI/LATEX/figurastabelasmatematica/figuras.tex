\documentclass[11pt,openright,twoside]{report}
\usepackage[utf8]{inputenc}
\usepackage{hyperref}
\usepackage{graphicx}

\begin{document}

\subsection{Figura}
\label{figure}
\begin{figure}[h]
\centerline{\fbox{Conteúdo da figura, pode ser texto, imagem, etc.}}
\caption{Legenda da figura}
\end{figure}

\pagebreak

\subsection{Tabela}
\label{tabel}
\begin{table}[htp]
\caption{Exemplo de uma tabela}
\centerline{Conteúdo de uma tabela}
\end{table}

\pagebreak

\subsection{Referências a partes do texto}
\label{refs.Section}
Qualquer identificador numérico usado num documento {\LaTeX}, seja ele de parte, capítulo, secção, lista numerada, figura, tabela, e outros, pode ser usado no texto através dos comandos ...

\pagebreak

\ref{figure}
\ref{tabel}
\ref{refs.Section}

\pagebreak

\begin{tabular}{|l||c|r|}
%
\hline
& Temperatura & Humidade \\
Cidade & (\textordmasculine C) & (perc.) \\ \hline\hline
Aveiro & {\large 10} & {\large 90} \\ \hline
Lisboa & {\tiny 13} & {\tiny 84} \\ \hline
Porto & \textbf{9} & \textbf{89} \\ \hline
%
\end{tabular}

\pagebreak

a expressão $$y = ax^2 + bx + c$$ é a forma geral da equação de 2º grau
\\\\\\
a expressão $y = ax^2 + bx + c$ é a forma geral da equação
de 2º grau
\\\\\\
a expressão
\[
y = ax^2 + bx + c
\]
é a forma geral da equação de 2º grau
\\\\\\
a expressão \ref{2ordem.eq}
\begin{equation}
y = ax^2 + bx + c \label{2ordem.eq}
\end{equation}
é a forma geral da equação de 2º grau
\\\\\\
Perímetro e área do círculo:
\begin{eqnarray}
P = 2\pi r \\
A = \pi r^2
\end{eqnarray}
\\\\\\
$y = \sum_{i=0}^{i=n}{x_{i}^{2n}}$
$$y = \sum_{i=0}^{i=n}{x_{i}^{2n}}$$
\\


\end{document}